%% ==============
\chapter{Prospect}
\label{ch:Prospect}
%% ==============

We have demonstrated powerful results for PNEE in chapter~\ref{ch:Evaluation}. With speedups of around 50-100x in scenes with merely a couple thousand light sources, PNEE proves to be a great choice. A thousand light sources are well within the scope of typical real world production. For rather extreme cases with around a million light sources, results would be amplified. With hardly any drawbacks, we found no peril in using \textit{Cdfgrid} as a standard even for low complexity scenes. Direct comparison to very recent work from \textcite{Vevoda:2018:BOR} and \textcite{Estevez} would be highly interesting.

 We have already discussed some improvements and future work in section~\ref{sec:futhercons}. Mainly improvements for adaptive paramentrization are substantial for increasing the ease of use. Adding Multiple Importance Sampling with adaptive importance weights might also be a valuable addition in the future. Exploring the possibility to extend \textit{Cdfgrid} for adaptive LOD with an Octree was discussed in section~\ref{ch:octree}. Improvements on importance sampling the initial photon distribution, as discussed in section~\ref{ch:photonimportancesample}, might also be worthwhile to explore. We examined several interpolation and approximation schemes in section~\ref{ch:interpolation}, but there is still plenty of research that can be done.\todo{add some research} It has been an exciting task to adapt and optimize traditional interpolation schemes to CDFs and our specific needs. 

Our construction of interpolated CDFs (section~\ref{sec:intcdf}) and sparse CDFs (section~\ref{sec:sparse}) have proven to be effective additions and may be of value for other research, too. The general idea of PNEE can be ported to other ray tracers like MLT or BDPT. Also, photon mapping can be extended with PNEE, essentially for free, since photons are shot and stored anyhow. In an interactive context, PNEE might be explored but might be tricky due to the preprocess. In the case where only the camera will be moved, PNEE is particularly suited, as the photons and data structures can be reused. Lastly, the idea of PNEE is potentially extendable to path guiding (see e.g. \parencite{DBLP:journals/cgf/MullerGN17}), where NEE can be regarded as just the last step in a series of subproblems.
