\chapter*{Abstract}

Modern photorealistic media production faces an increasing number of light sources to produce natural-looking images. A very significant percentage of processing time goes towards casting shadow rays to calculate the direct lighting term with Next Event Estimation. Naively sampling many light sources in a path tracer quickly becomes impractical, as our evaluation shows. We present a novel technique to importance sample many light sources called Photon-based Next Event Estimation (PNEE). In a preprocess,  similar to photon mapping, photons are scattered from all light sources into the scene. These photons are used to build cumulative distribution functions (CDF) that reflect the importance of the light sources to reduce the variance of Monte Carlo Integration. We evaluate several storage options and data structures for efficient lookups within the integrator. We introduce and utilize two novel data structures to store CDFs: sparse CDFs and interpolated CDFs. To mitigate variance edges and artifacts, we present several interpolation and approximation techniques. We evaluate several variants of PNEE against naive NEE as well as PBRTs recent implementation, and demonstrate that even for moderately complex scenes a 50-100x speedup can be achieved.