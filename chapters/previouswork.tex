%% ==============
\chapter{Related Work}
\label{ch:Prev}
%% ==============

We introduced academic fundamentals that we base our work on in the previous section. In this chapter, we further evaluate both past and recent work that focuses specifically on handling many light sources. Some work of late explicitly targets Next Event Estimation, while similar problems can also be found in work about Instant Radiosity \parencite{keller1997instant, Walter2005LightcutsAS, dachsbacher2014scalable}. Due to the large number of virtual point lights, several techniques had been developed to scale sublinearly. We have to be cautious when taking these techniques into account, as one of our main goals is to remain unbiased; most work on Instant Radiosity is dependent on biased approximations.

\paragraph{Early work}

Early work on this topic from \textcite{ward1994adaptive} sorts light sources by power and tests them sequentially up until a cutoff.  \textcite{Shirley:1996:MCT:226150.226151} later improve this by introducing the construction of probability distribution functions (PDF) for simple area light shapes. Sampling within an area light is a problem on its own but is well understood for basic shapes. They then describes how those PDFs can be combined to inherit multiple light sources. \citeauthor{Shirley:1996:MCT:226150.226151} also discuss weighting the sampling probability of the light sources uniformly and based on energy, which produce reasonable results in simple scenes but quickly breaks down with many light sources and occlusion. This is, in essence, the technique we later refer to as \textit{power} in chapter~\ref{ch:Evaluation} discussions. They then further propose a technique to spatially divide the scene with an octree to account for more localized influences. Still, these calculations remain based on light power, angle, and spatial distance and fail to attribute for any kind of occlusion.

\textcite{DBLP:conf/vmv/KellerW00} extend photon mapping to utilize its already existing photon map for importance sampling. The proposed algorithm only roughly classifies light sources into two importance groups and then importance samples accordingly. For the generation single photons are used; we later show in chapter~\ref{ch:Evaluation} that this is not feasible, despite introducing several improvements in chapter~\ref{ch:PNEE}. \textcite{DBLP:conf/rt/WaldBS03} propose an approach for interactive applications. A crude path tracer preprocessing step is utilized. With a downsized resolution light, importance PDFs are precached in screen space. 

\paragraph{Lightcuts}

\textcite{DBLP:journals/cgf/PaquettePD98} are one of the first to introduce a light hierarchy (tree) which can be used to move down in detail until approximated criteria are met. This concept is extended upon and adapted to Instant Radiosity by \textcite{Walter2005LightcutsAS} in their landmark paper introducing Lightcuts. A Lighttree hierarchically combines the light sources of a scene. The leaves are light sources which get pooled into clusters by position and orientation. A Lightcut, in turn, is a set of nodes such that a path from any leaf to the root will contain exactly one node from the cut. We discuss how Lightcuts might be applied to PNEE in section~\ref{sec:lightcutd}.

More recent work from \textcite{Estevez} is inspired by Lightcuts and ports some ideas to unbiased Next Event Estimation for many lights. Just like in Lightcuts, a Lighttree is constructed. Later, to choose a light source, a cut is calculated by moving down the clusters as long criteria for position and normal cone orientation are met. All lights found below the cut are then sampled uniformly or based on energy.

\paragraph{Grid caching}

\textcite{Vevoda} propose a technique which also utilizes a Lighttree and Lightcuts but combines it with a world space cache. The scene is subdivided into a regular grid where each cell holds a cut of the Lighttree. Similar to \citeauthor{Estevez}, this cut can be used to importance sample given light sources. Noteworthy is that a cut is lazily calculated only when a shading point within an empty cell is rendered, but can later be reused for all shading points within the same grid cell.

By coincidence, PBRT \parencite{pbrt}, the renderer we use throughout this work, recently committed a similar (yet undocumented) technique to its Github repository. It is not mentioned in edition three of the book but likely will appear in its next edition. Comments indicate that such a technique was crucial to be able to render scenes with many lights like Zero-Day, which we also use for comparison in chapter~\ref{ch:Evaluation}. \citeauthor{pbrt} do not use Lightcuts. Instead, usual PDFs are constructed, but the algorithm for lazy world space caching in a grid is very similar to \citeauthor{Vevoda}'s technique.

To elaborate, whenever a shading point is rendered within a grid cell that is not yet cached, 128 random points within the cell are sampled and deterministically connected to all light sources (a random point for area lights). Based on the position, distance, and orientation of the light, an average importance weight for any light source is calculated and a PDF is constructed. In its current state, occlusion is wholly ignored. However, a comment in the code indicates, that there is or was an attempt to include occlusion; we discuss this further in section~\ref{sec:pbrtoccl}. We continue to refer to the technique as it is internally referred to: \textit{Spatial}.

\paragraph{Reinforcement learning}

\parencite{DBLP:journals/corr/DahmK17} utilize machine learning in their recent work which details how light transport paths are more generally learned by a $\mathcal{Q}$-function with reinforcement learning. The $\mathcal{Q}$-functions are distributed in screen space and form a Voronoi diagram which guides sampling.

\paragraph{Outlook}

Just recently before the publication of this work \textcite{Vevoda:2018:BOR} present an extension of their previous work. A Bayesian online learning algorithm is utilized to gradually learn occlusion into the initially naively constructed Lightcut cache. We continue our discussion on this work in section~\ref{sec:pbrtoccl}.