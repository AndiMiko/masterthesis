%% ==============
\chapter{Related Work}
\label{ch:Prev}
%% ==============

We introduced many academic fundamentals we base our work on in the previous section. In this section, we further evaluate past and recent work that particularly focuses on handling many light sources. Some recent work explicitly targets Next Event Estimation, while similar problems can be found on work about Instant Radiosity \parencite{keller1997instant, Walter2005LightcutsAS, dachsbacher2014scalable}. Due to the large number of virtual point lights several techniques had been developed to scale sublinearly. We have to consider these techniques with caution, as one of our main goals is to stay unbiased, while most work on Instant Radiosity is dependent on biased approximations.

\paragraph{Early work}

Early work on this topic from \textcite{ward1994adaptive} sorts light sources by power and tests them sequentially up until a cutoff. Later \textcite{Shirley:1996:MCT:226150.226151} improves this by introducing the construction of probability distribution functions (PDF) for simple area light shapes. Sampling within an area light is a problem on its own but is well understood for basic shapes. Further, he describes how those PDFs can be combined to inherit multiple light sources. \citeauthor{Shirley:1996:MCT:226150.226151} discuss weighting the sampling probability of the light sources uniformly and energy based which produces reasonable results in simple scenes but quickly breaks down with many light sources and occlusion. This is essentially the technique we later refer to as \textit{power} in chapter~\ref{ch:Evaluation} discussions. They further propose a technique to spatially divide the scene with an octree to account for more localized influences. Still, calculations remain based on light power, angle and spatial distance and fail to attribute for any kind of occlusion.

\textcite{DBLP:conf/vmv/KellerW00} extends photon mapping to utilize its already existing photonmap for importance sampling. The proposed algorithm only makes a rough classification into two importance groups and then importance samples accordingly. For the generation single photons are used, we later show in chapter~\ref{ch:Evaluation} that this is not feasible even though we introduced several improvements in chapter~\ref{ch:PNEE}. \textcite{DBLP:conf/rt/WaldBS03} proposes an approach for interactive applications. A crude path tracer preprocessing step is utilized. With a downsized resolution light importance PDFs are precached in screen space. 

\paragraph{Lightcuts}

\textcite{DBLP:journals/cgf/PaquettePD98} are one of the first to introduce a light hierarchy (tree) which can be used to move down in detail until approximated criteria are met. This concept is extended upon and adapted to Instant Radiosity by \textcite{Walter2005LightcutsAS} in its landmark paper introducing lightcuts. A lighttree hierarchically combines light sources of a scene. The leaves are light sources which get pooled into clusters by position and orientation. A lightcut, in turn, is a set of nodes such that a path from any leaf to the root will contain exactly one node from the cut. We discuss how lightcuts might be applied to PNEE in section~\ref{sec:lightcutd}.

More recent work from \textcite{Estevez} is inspired by lightcuts and ports some ideas to unbiased Next Event Estimation for many lights. Just like in lightcuts a lighttree is constructed. Later, to choose a light source, a cut is calculated by moving down the clusters as long criteria for position and normal cone orientation are met. All lights below the found cut are then sampled uniformly or energy based.

\paragraph{Grid caching}

\textcite{Vevoda} propose a technique which utilizes a light tree and cuts as well but combines it with a world space cache. The scene is subdivided into a regular grid where each cell does hold a cut of the light tree. Similar to \citeauthor{Estevez} this cut can be used to importance sample given light sources. The kicker is that a cut is calculated lazily only when a shading point within an empty cell is rendered, but later can be reused for all shading points within the same grid cell.

By coincidence, the renderer PBRT \parencite{pbrt} which we use throughout this work recently committed a similar yet undocumented technique to its Github repository. It is not mentioned in edition three of the book but likely will appear in its next edition. Comments indicate that such a technique was crucial to be able to render scenes with many lights like Zero-Day, which we also use for comparison in chapter~\ref{ch:Evaluation}. \citeauthor{pbrt} do not use lightcuts instead usual PDFs are constructed, but the algorithm for lazy world space caching in a grid is very similar to \citeauthor{Vevoda}s technique.

In more detail, whenever a shading point is rendered within a grid cell that is not yet cached 128 random points within the cell are sampled and deterministically connected to all light sources (a random point for area lights). Based on the position, distance and orientation of the light an average importance weight for any light source is calculated and a PDF is constructed. In its current state occlusion is wholly ignored, but a comment in the code indicates that there is an attempt to include occlusion, we further discuss this in section~\ref{sec:pbrtoccl}. We continue to refer to the technique just like it is internally referred to: \textit{spatial}.

\paragraph{Reinforcement learning}

Another recent work from \parencite{DBLP:journals/corr/DahmK17} utilizes machine learning. It more generally learns light transport paths by learning a $\mathcal{Q}$-function with reinforcement learning. The $\mathcal{Q}$-functions are distributed in screen space and form a Voronoi diagram which guides sampling.

\paragraph{Outlook}

Just recently before the publication of this work \textcite{Vevoda:2018:BOR} present an extension of their previous work. A Bayesian online learning algorithm is utilized to gradually learn occlusion into the initially naively constructed lightcut cache.  \todo{KIst mittlerweile noch etwas neues veröffentlicht?}