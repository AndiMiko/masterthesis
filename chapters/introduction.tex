%% ==============================
\chapter{Introduction}
\label{ch:Introduction}
%% ==============================

There are generally two methods to render a picture from a scene description: rasterization and ray tracing. With rasterization you start with the objects and splat them onto the view frustum. This performs very well and with various extending techniques very good looking, but not necessary physically correct, results can be achieved.  Rasterization is very common for real-time applications and graphic card pipelines are highly optimized for it. On the other hand ray tracing is usually considered slow and runs on the CPU, so use cases are more limited, but actually tracing light paths from light sources to the lense is a necessity to allow for physically correct and unbiased estimation of a scenes light transport. \unsure{Gibt es wirklich nur 2 methoden? Es gibt auch auf ray tracing optimierte Grafik Karten? Viele Kleinigkeiten, sind alle getroffenen Annahmen korrekt?}

Because of the quite intricate nature of light itself (e.g. wave-particle dualism) it is not really possible to solve the light transport equation with a basically infinite number of dimensions. The solution is to explore as many, hopefully important, light paths as possible with each decision for the path construction done probabilistically. For each event that gets chosen by a random variable the result is later divided by the probability of its occurrence. More specifically the continuous integrals are estimated with the Monte Carlo integration, allowing the estimation of a continuous function with only a limited number of samples. We introduce these concepts in more detail in chapter~\ref{ch:Fundamentals}.

When constructing the paths there are many choices and considerations one can take. Should you start at the eye or the light source, or both and later connect the path segments? How to compute the reflection on a surface? How does the medium affect the current path and refraction? Can you adjust the probabilities based on a preprocess or guide paths based on precending path calculations?
These questions spark a variety of techniques and sub-techniques, our Photon-based Next Event Estimation (PNEE) is one of them. Acknowledging the existence and possible suitability of techniques like Bidirectional Path-tracing (BDPT), Metropolis light transport and others, throughout this work we focus our discussion on path tracing.

With path tracing each path is started at the eye and only one refraction is considered at each intersection. The depth of the path is commonly limited by Russian Roulette (RR). For each pixel a multitude of paths are started from random positions within the pixel (providing free Anti-aliasing). Aside from the single main path which covers indirect lighting it is very effective to use Next Event Estimation (NEE) at every intersection point. NEE does estimate the direct lighting by deterministically connecting each intersection point to a light source. The rendering equation can be split up into a direct and indirect lighting part to enable this without introducing bias (see section~\ref{sec:NEE}).

Modern photorealisitic rendering is commonly done by path tracers. With an ever expanding complexity of modelled scenes the presence of many light sources greatly contributes to a natural and appealing image. Path tracers utilize Next Event Estimation to deterministically connect ray intersections (shading points) with light sources. This process is a critical step in modern path tracers and takes up to 50\% of today's rendering time, partially because calculating the visibility term is costly. As sampling all lights quickly becomes too expensive for scenes with more then a few dozen lights, sampling only one or some lights per intersection is a common approach to greatly reduce the rendering time of scenes with a lot of occlusion respectively localized light influences. In an attempt to reduce the variance it is apparent that you want to sample light sources with a higher contribution with a higher probability and try to spare any unnecessary work of sampling occluded light sources. An important aspect is to stay unbiased during this process. We investigate current techniques and propose a new solution to this problem.